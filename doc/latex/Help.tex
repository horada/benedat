\documentclass[12pt]{article}
\usepackage[utf8]{inputenc}
\usepackage{czech}
\usepackage{listings}
\usepackage[top=2cm, bottom=3cm]{geometry}

\lstnewenvironment{kodConf}
    {}
    {}

\newcommand{\tl}[1]{\emph{#1}}

    \begin{document}
\author{Daniel Horák}
\title{Nápověda k programu BeneDat}
\maketitle
\tableofcontents


\section{Úvod}
Program BeneDat je určen ke zjednodušení výpočtu ceny pro klienty za využité služby občanského
sdružení. V současné verzi umožňuje zaznamenávat dny a hodiny využívání odlehčovací služby a s tím
související dopravu klienta. Z těchto hodnot dojde při generování sestavy k výpočtu celkové ceny,
vygenerování pdf souboru se sestavou (zároveň sloužící jako příjmový pokladní doklad) a xml souboru
který je možno použít k exportu dat do účetního programu Pohoda.
\section{Práce s programem BeneDat}
\subsection{Před prvním spuštěním}
Instalace programu je popsána na jiném místě, zde pouze zmíním některé věci které je dobré (spíše
nutné) nastavit v konfiguračním souboru. Pokud jste ještě program BeneDat nespustili, konfigurační
soubor ještě pravděpodobně není vytvořen. Spusťte tedy program BeneDat a hned jej zase můžete
ukončit. V hlavním adresáři programu (tam kde se nachází například soubor benedat.py) by jste měli
nalézt soubor benedat.conf se zhruba takovýmto obsahem:
\begin{verbatim}
  kod_cinnosti="odl"
  ico="0000"
  pokladna="HP"
  kod_strediska="0011-vl.zd"
  kod_predkontace="6023-odleh"
  poznamka_k_exportu="Převod dat z programu BeneDat"
  kontrola_duplicity="true"
  kod_cleneni_dph="nonSubsume"
  otevreny_soubor="None"
  aplikace="BeneDat"
\end{verbatim} 
Zde je potřeba pozměnit některé volby pro správné fungování exportu do programu
Pohoda.
\subparagraph{kod\_cinnosti} kód činnosti jaký bude přiřazen dokladu
\subparagraph{ico} ičo organizace (pokud není uvedeno správě, při exportu do Pohody dostaneme
chybové hlášení ve smyslu \emph{tato položka není určena pro tuto účetní jednotku}.
\subparagraph{pokladna} kód pokladny
\subparagraph{kod\_strediska} kód střediska
\subparagraph{kod\_predkontace} kód předkontace
\subparagraph{poznamka\_k\_exportu} poznámka uvedená u exportu
%\subparagraph{kontrola\_duplicity}
\subparagraph{kod\_cleneni\_dph} kód členění dph
\subparagraph{otevreny\_soubor} naposledy otevřený soubor (není potřeba upravovat -- nastavuje se
automaticky)
\subparagraph{aplikace} hodnota který bude u exportu do Pohody uvedena v kolonce aplikace 
Editaci souboru můžeme provést libovolným jednoduchým textovým editorem (například PSPad, notepad a
podobně).


\subsection{První spuštění}
\subsection{Vytvoření nové -- otevření stávající databáze}
Stiskem příslušného tlačítka \tl{Nový} nebo \tl{Otevřít} vytvoříme/otevřeme databázový soubor. Celá
databáze je uložena v jednom souboru s příponou db.
\subsection{Základní nastavení}
Při vytvoření nové databáze případně při změně některých údajů je potřeba provést nastavení
konkrétní databáze. Toto nastavení je dostupné po rozbalení položky \tl{Pokročilé} v hlavním okně
programu a stiskem příslušného tlačítka \tl{Nastavení}.


\end{document}
